% 通常IRL(拡張IRLじゃない方)の特徴量定義
% State Features: 9次元, Action Features: 4次元(時間経過除外版)

\documentclass{article}
\usepackage[utf8]{inputenc}
\usepackage{booktabs}
\usepackage{array}
\usepackage{geometry}
\geometry{margin=1in}

\title{通常IRL特徴量定義}
\author{}
\date{}

\begin{document}

\maketitle

\section{状態特徴量(State Features): 9次元}

通常IRLでは、時間特徴量を除外して9次元の状態特徴量を使用しています。

\begin{table}[h]
\centering
\caption{状態特徴量(9次元)}
\begin{tabular}{@{}clll@{}}
\toprule
\textbf{No.} & \textbf{特徴量名} & \textbf{説明} & \textbf{定義} \\
\midrule
1 & \texttt{experience\_days} & 経験日数 & 初回活動日から現在までの日数(年単位に正規化: /365.0) \\
2 & \texttt{total\_changes} & 累積変更数 & これまでの総コミット数(/100.0で正規化) \\
3 & \texttt{total\_reviews} & 累積レビュー数 & これまでの総レビュー数(/100.0で正規化) \\
4 & \texttt{project\_count} & 参加プロジェクト数 & 参加しているプロジェクトの数(/10.0で正規化) \\
5 & \texttt{recent\_activity\_frequency} & 最近の活動頻度 & 過去30日間の活動日数 / 30(0-1範囲) \\
6 & \texttt{avg\_activity\_gap} & 平均活動間隔 & 活動タイムスタンプ間の平均日数(月単位: /30.0) \\
7 & \texttt{activity\_trend} & 活動トレンド & 最近30日 vs 過去30-60日の活動量比較(increasing=1.0, stable=0.5, decreasing=0.0) \\
8 & \texttt{collaboration\_score} & 協力スコア & レビュー・マージ等の協力活動の比率(0-1範囲) \\
9 & \texttt{code\_quality\_score} & コード品質スコア & test/doc/refactor/fixの頻度(0-1範囲) \\
\bottomrule
\end{tabular}
\end{table}

\section{行動特徴量(Action Features): 4次元}

通常IRLでは、時間特徴量を除外して4次元の行動特徴量を使用しています。

\begin{table}[h]
\centering
\caption{行動特徴量(4次元)}
\begin{tabular}{@{}clll@{}}
\toprule
\textbf{No.} & \textbf{特徴量名} & \textbf{説明} & \textbf{定義} \\
\midrule
1 & \texttt{intensity} & 行動の強度 & レビューリクエストへの応答の強さ(変更行数ベース: (追加行数+削除行数)/(ファイル数×50)) \\
2 & \texttt{message\_quality} & メッセージの質 & レビューリクエストのメッセージの質(fix, improve, test等のキーワードが含まれているか) \\
3 & \texttt{collaboration} & 協力度 & レビューを通じた他の開発者との協力の程度(レビュー依頼への応答率等) \\
4 & \texttt{latency} & 応答遅延 & レビューリクエストから応答までの時間(日数) \\
\bottomrule
\end{tabular}
\end{table}

\section{特徴量の詳細説明}

\subsection{状態特徴量の計算方法}

\begin{itemize}
\item \textbf{経験日数}: 初回活動日から現在までの日数
\item \textbf{活動頻度}: 過去30日間の活動日数 / 30
\item \textbf{活動間隔}: 活動タイムスタンプ間の平均日数
\item \textbf{活動トレンド}: 最近30日 vs 過去30-60日の活動量比較
\item \textbf{協力スコア}: レビュー・マージ等の協力活動の比率
\item \textbf{コード品質スコア}: test/doc/refactor/fixの頻度
\end{itemize}

\subsection{行動特徴量の計算方法}

\begin{itemize}
\item \textbf{強度}: レビューリクエストへの応答の強さ(変更行数ベース)
\item \textbf{メッセージの質}: レビューリクエストのメッセージの質(fix, improve, test等のキーワードが含まれているか)
\item \textbf{協力度}: レビューを通じた他の開発者との協力の程度(レビュー依頼への応答率等)
\item \textbf{応答遅延}: レビューリクエストから応答までの時間(日数)
\end{itemize}

\section{拡張IRLとの比較}

\begin{table}[h]
\centering
\caption{通常IRL vs 拡張IRL}
\begin{tabular}{@{}lcc@{}}
\toprule
\textbf{項目} & \textbf{通常IRL} & \textbf{拡張IRL} \\
\midrule
状態特徴量 & 9次元 & 32次元 \\
行動特徴量 & 4次元 & 9次元 \\
特徴量の種類 & 基本的な特徴量のみ & 多期間・負荷・相互作用等を追加 \\
実行速度 & 速い & やや遅い \\
メモリ使用量 & 少ない & やや多い \\
\bottomrule
\end{tabular}
\end{table}

\end{document}
